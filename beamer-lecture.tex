%%%%%%%%%%%%%%%%%%%%%%%%%%%%%%%%%%%%%%%%%%%%%%%%%%%%%%%%%%%%
% My Beamer for Lecture by Y.Chubachi (2016)
%%%%%%%%%%%%%%%%%%%%%%%%%%%%%%%%%%%%%%%%%%%%%%%%%%%%%%%%%%%%

% Beamerのマニュアルは次のコマンドで参照できる
%   $ texdoc beamer

%%% Beamerテーマの指定

% Beamerのテーマと配色のマトリクス
%   Beamer Theme Matrix
%   - https://www.hartwork.org/beamer-theme-matrix/

\usetheme{default}
% \usetheme{Szeged} % 横線
% \usetheme{Madrid}
% \usetheme{Goettingen} % サイドバーの色が変わらない
% \usetheme{Hannover} % 左サイドバー
% \usetheme[hideothersubsections]{Berkeley}
% \usetheme{AnnArbor}

%%% 基調となる色の設定

% 色の見本は次のコマンドで調べることができる
% $ texdoc xcolor # dvipsnamesの箇所

% FF6410 = AIITオレンジ
\definecolor{LogoColor}{HTML}{FF6410} % ロゴの色(AIITオレンジ)
\definecolor{MainColor}{HTML}{E66E1E} % 基調となる色
\definecolor{TitleFg}{named}{MainColor} % タイトルの文字色
\definecolor{TitleBg}{HTML}{FFDEB9} % タイトルの背景色
\definecolor{MenuBg}{HTML}{FFF4E8} % メニューの背景色

\usecolortheme{default}
\setbeamercolor{normal text}{fg=black,bg=white}
\setbeamercolor{alerted text}{fg=red}
\setbeamercolor{example text}{fg=green!50!black}

\setbeamercolor{structure}{fg=MainColor}

\setbeamercolor{palette primary}{fg=LogoColor, bg=TitleBg}
\setbeamercolor{palette secondary}{fg=LogoColor,bg=MenuBg}
\setbeamercolor{palette tertiary}{fg=LogoColor, bg=TitleBg}
\setbeamercolor{palette quaternary}{fg=green}

% \setbeamercolor{palette sidebar primary}{fg=red}
% \setbeamercolor{palette sidebar secondary}{fg=green}
% \setbeamercolor{palette sidebar tertiary}{fg=blue}
% \setbeamercolor{palette sidebar quaternary}{fg=black}

% \setbeamercolor{background canvas}{bg=purple}
\setbeamercolor{titlelike}{fg=white, bg=LogoColor}
% \setbeamercolor{sidebar}{bg=black}

% これ以外の設定は
%   /usr/local/texlive/2016/texmf-dist/tex/latex/beamer/themes/color/beamercolorthemedefault.sty
% などを参照

%%%%%%%%%%%%%%%%%%%%%%%%%%%%%%%%%%%%%%%%%%%%%%%%%%%%%%%%%%%%
%% ナビゲーション
\setbeamertemplate{navigation symbols}{} % ナビゲーションシンボルを消す
\setbeamertemplate{footline}[frame number] % フッターはスライド番号のみ

%% アウターテーマ
\useoutertheme{infolines}
% \useoutertheme[height=1.5cm, width=2cm, right]{sidebar}
% \setbeamercolor{frametitle}{fg=white} % heightを設定すると黒になるのを修正

%% インナーテーマ
% \useinnertheme{circles} % 箇条書きをシンプルに
% \useinnertheme{rounded}
% \setbeamertemplate{blocks}[rounded] % 影を消す(オプションで消せない?)

% フォントテーマ
\usefonttheme{structurebold}
\renewcommand{\kanjifamilydefault}{\gtdefault}  % 和字をゴシック体に

% フレームのカスタマイズ
\setbeamerfont{frametitle}{size=\large}
\setbeamertemplate{itemize/enumerate body begin}{\normalsize}
\setbeamertemplate{itemize/enumerate subbody begin}{\normalsize}

% 第2階層ごとに表題を表示
\AtBeginSection[]{
  \lecture{\insertsection}
}

\AtBeginLecture{
  \begin{frame}
    \date{\insertlecture}
    \maketitle
  \end{frame}
}

% 第3階層ごとに目次を表示
\AtBeginSubsection[]{
  \begin{frame}{\inserttitle}
    \tableofcontents[currentsubsection,
    sectionstyle=show/hide,
    subsectionstyle=show/shaded/hide]
  \end{frame}
}

\setbeamertemplate{section in toc}[ball unnumbered]
\setbeamertemplate{subsection in toc}[square]

% 参考
% - Beamer — Tasuku Soma's webpage
% http://www.opt.mist.i.u-tokyo.ac.jp/~tasuku/beamer.html#tips
