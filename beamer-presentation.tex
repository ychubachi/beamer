%%%%%%%%%%%%%%%%%%%%%%%%%%%%%%%%%%%%%%%%%%%%%%%%%%%%%%%%%%%%
% My Beamer for Lecture by Y.Chubachi (2016)
%%%%%%%%%%%%%%%%%%%%%%%%%%%%%%%%%%%%%%%%%%%%%%%%%%%%%%%%%%%%

%%% Beamerのマニュアルは次のコマンドで参照できる
%   $ texdoc beamer

%%% Beamerテーマの指定

% Beamerのテーマと配色のマトリクス
%   Beamer Theme Matrix
%   - https://www.hartwork.org/beamer-theme-matrix/

\usetheme{default}

%%% 基調となる色の設定

% 色の見本は次のコマンドで調べることができる
% $ texdoc xcolor # dvipsnamesの箇所
\definecolor{lightgreen}{HTML}{ABE3AE}
\definecolor{green}     {HTML}{3FB83D}
\definecolor{lightbrown}{HTML}{D0AC71}
\definecolor{brown}     {HTML}{773E28}

\usecolortheme{default}
\setbeamercolor{normal text}{fg=black,bg=white}
\setbeamercolor{alerted text}{fg=brown}
\setbeamercolor{example text}{fg=green!50!black}

% \setbeamercolor{structure}{fg=MainColor}

% \setbeamercolor{palette primary}{fg=LogoColor, bg=TitleBg}
% \setbeamercolor{palette secondary}{fg=LogoColor,bg=MenuBg}
% \setbeamercolor{palette tertiary}{fg=LogoColor, bg=TitleBg}
% \setbeamercolor{palette quaternary}{fg=green}

\setbeamercolor{structure}{fg=brown}
% \setbeamercolor{background canvas}{bg=purple}
\setbeamercolor{titlelike}{}

\setbeamercolor{palette primary}{fg=red}
\setbeamercolor{palette secondary}{fg=red,bg=green} % SBの右上の四角も
\setbeamercolor{palette tertiary}{fg=blue}
\setbeamercolor{palette quaternary}{fg=yellow}


%%% アウターテーマ
%% サイドバーの設定
\useoutertheme[height=1.5cm, width=2cm, right]{sidebar} % サイドバーの配置
\setbeamercolor{sidebar}{bg=lightgreen} % 背景
\setbeamercolor{palette sidebar primary}{fg=red,bg=yellow} % 未使用?
\setbeamercolor{palette sidebar secondary}{fg=brown,bg=green} % セクション
\setbeamercolor{palette sidebar tertiary}{fg=black} % 著者
\setbeamercolor{palette sidebar quaternary}{fg=brown} % タイトル

%%% インナーテーマ
\useinnertheme{circles} % 箇条書きをシンプルに
\useinnertheme{rounded}
\setbeamertemplate{blocks}[rounded] % 影を消す(オプションで消せない?)

%%% ナビゲーション
\setbeamertemplate{navigation symbols}{} % ナビゲーションシンボルを消す
\setbeamertemplate{footline}[frame number] % フッターはスライド番号のみ

%%% フォントテーマ
\usefonttheme{structurebold}
\renewcommand{\kanjifamilydefault}{\gtdefault}  % 本文の和字もゴシック体に

%%% フレームのカスタマイズ
%% フレームタイトルのサイズを少々小さめに
\setbeamerfont{frametitle}{size=\large}
\setbeamertemplate{itemize/enumerate body begin}{\normalsize}
\setbeamertemplate{itemize/enumerate subbody begin}{\normalsize}

%%% 節ごとの目次フレーム
\AtBeginSection[]{
  \begin{frame}
    \frametitle{目次}
    \tableofcontents[currentsection,currentsubsection]
  \end{frame}
}
